\section{DYNAMICAL TRANSLATIONAL MODEL OF a QUADROTOR}

In this section, we delve into the dynamical translational model designed for a quadcopter drone, which underpins an advanced Linear Quadratic Gaussian (LQG) control scheme. This model intricately captures the drone's motion, factoring in drag forces to accurately reflect its behavior under various flight conditions and is crucial for effective noise handling in control systems, enabling precise control despite measurement noise.

\subsection{Model Rationale and Design}

The selected model is integral for the Extended Kalman Filter (EKF) implementation, essential for estimating the drone's velocity amidst noise. The velocity estimates are derived from the onboard quadrotor Inertial Measurement Unit (IMU) and position inputs from an OptiTrack system, with the latter subjected to artificially added noise to simulate real-world measurement inaccuracies.

\subsection{Detailed Model Functions and Their Significance}

\subsubsection{World Frame Dynamics}

The drone's motion in the world frame is described by the velocities \( \dot{x}_w \), \( \dot{y}_w \), and \( \dot{z}_w \), obtained through a transformation of the body frame velocities, considering the drone's orientation:

\begin{align}
\dot{x}_w &= \cos(\psi) \cdot \dot{x}_b - \sin(\psi) \cdot \dot{y}_b, \\
\dot{y}_w &= \sin(\psi) \cdot \dot{x}_b + \cos(\psi) \cdot \dot{y}_b, \\
\dot{z}_w &= \dot{z}_b,
\end{align}

where \( \psi \) denotes the yaw angle. This transformation is pivotal for integrating position data, albeit noisy, from the OptiTrack system in the world frame with the drone's estimated velocities from the body frame via the IMU, enhancing state estimation accuracy crucial for the EKF.

\subsubsection{Body Frame Dynamics}

The accelerations in the body frame, reflecting the drone's response to control inputs and environmental factors, are given by:

\begin{align}
\ddot{x}_b &= -\frac{K_{vx}}{m} \cdot \dot{x}_b + \frac{t}{m} \cdot \sin(\theta), \\
\ddot{y}_b &= -\frac{K_{vy}}{m} \cdot \dot{y}_b + \frac{t}{m} \cdot \sin(\phi), \\
\ddot{z}_b &= -\frac{K_{vz}}{m} \cdot \dot{z}_b + \frac{t}{m} \cdot \cos(\theta) \cdot \cos(\phi) - g,
\end{align}

where \( K_{vx} \), \( K_{vy} \), and \( K_{vz} \) represent the drag coefficients, \( m \) the drone's mass, \( t \) the thrust, \( \theta \) and \( \phi \) the pitch and roll angles, and \( g \) the gravitational acceleration. These expressions are crucial for the state-space representation in the Linear Quadratic Regulator (LQR) for velocity tracking, enabling dynamic adjustment to the desired reference velocities.

\subsection{Model Linearization}
This subsection extends the dynamical translational model of a quadcopter drone to include the linearization process for control design, specifically targeting the implementation of a Linear Quadratic Regulator (LQR). The model's adaptation facilitates an optimized control strategy under the Linear Quadratic Gaussian (LQG) framework, enabling precise velocity control despite the presence of measurement noise and other practical challenges.

The nonlinear dynamical model, capturing the drone's motion with respect to drag forces and orientation-influenced accelerations, undergoes linearization to enable the application of LQR for velocity control. This process involves two key approximations: the small angle approximation and the thrust-weight balance.

\subsubsection{Small Angle Approximation}

For small pitch (\(\theta\)) and roll (\(\phi\)) angles, the trigonometric functions are approximated as \(\sin(\theta) \approx \theta\) and \(\sin(\phi) \approx \phi\), while \(\cos(\theta)\) and \(\cos(\phi)\) are approximated as 1. This simplification yields the following linearized equations for the body frame accelerations:

\begin{align}
\ddot{x}_{body} &= -\frac{K_{vx}}{m} \cdot \dot{x}_{body} + g \cdot \theta, \\
\ddot{y}_{body} &= -\frac{K_{vy}}{m} \cdot \dot{y}_{body} + g \cdot \phi, \\
\ddot{z}_{body} + g &= -\frac{K_{vz}}{m} \cdot \dot{z}_{body} + \frac{1}{m} \cdot t.
\end{align}

\subsubsection{Thrust-Weight Balance}

Considering the thrust (\(t\)) to approximately equal the drone's weight (\(mg\)), facilitates a direct control impact on the \(z\)-axis acceleration, further simplifying the linearization.

\subsection{State-Space Representation for LQR Control}

The linearized model is represented in state-space form as follows, which is essential for the design and implementation of the LQR:

\begin{align}
A &= \begin{bmatrix} -\frac{K_{vx}}{m} & 0 & 0 \\ 0 & -\frac{K_{vy}}{m} & 0 \\ 0 & 0 & -\frac{K_{vz}}{m} \end{bmatrix}, \\
B &= \begin{bmatrix} g & 0 & 0 \\ 0 & g & 0 \\ 0 & 0 & \frac{1}{m} \end{bmatrix}, \\
C &= \begin{bmatrix} 1 & 0 & 0 \\ 0 & 1 & 0 \\ 0 & 0 & 1 \end{bmatrix}, \\
D &= \begin{bmatrix} 0 & 0 & 0 \\ 0 & 0 & 0 \\ 0 & 0 & 0 \end{bmatrix}.
\end{align}

The state matrix \(A\) describes the system dynamics, the input matrix \(B\) relates control inputs to the state derivatives, and the matrices \(C\) and \(D\) define the system's outputs. Here, the states are the body frame velocities \([\dot{x}_{body}, \dot{y}_{body}, \dot{z}_{body}]\), and the inputs are the control actions \([\theta, \phi, t]\).

\subsection{Control Design Objective}

The LQR is designed to minimize a cost function that includes terms for both the deviation of states from desired velocities and the control effort. This approach ensures that the drone's velocity converges asymptotically to the reference values generated by the outer control loop, optimizing performance in the presence of measurement noise and achieving robust velocity control.

The integration of the linearized model within an LQR framework exemplifies the advanced control strategy's capability to maintain precise control over the drone's motion, highlighting the model's utility and the control scheme's effectiveness in practical scenarios.


\subsection{Integration with LQG Control Scheme}

The clear delineation of the model into world and body frame dynamics facilitates a dual-layered control approach: leveraging the world frame dynamics for accurate state estimation via the EKF amidst noisy measurements, and employing the body frame dynamics for precise velocity control through the LQR. This strategic integration within an LQG framework exemplifies the control scheme's robustness, optimizing performance despite the challenges posed by measurement noise and environmental uncertainties.

The experimental validation with the \emph{Crazyflie} quadrotor underscores the efficacy of our model and the LQG control scheme, demonstrating remarkable precision in velocity control and robustness against measurement noise, thereby validating the proposed system's practical applicability and effectiveness.